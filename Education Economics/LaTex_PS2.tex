
% Default to the notebook output style

    


% Inherit from the specified cell style.




    
\documentclass[11pt]{article}

    
    
    \usepackage[T1]{fontenc}
    % Nicer default font (+ math font) than Computer Modern for most use cases
    \usepackage{mathpazo}

    % Basic figure setup, for now with no caption control since it's done
    % automatically by Pandoc (which extracts ![](path) syntax from Markdown).
    \usepackage{graphicx}
    % We will generate all images so they have a width \maxwidth. This means
    % that they will get their normal width if they fit onto the page, but
    % are scaled down if they would overflow the margins.
    \makeatletter
    \def\maxwidth{\ifdim\Gin@nat@width>\linewidth\linewidth
    \else\Gin@nat@width\fi}
    \makeatother
    \let\Oldincludegraphics\includegraphics
    % Set max figure width to be 80% of text width, for now hardcoded.
    \renewcommand{\includegraphics}[1]{\Oldincludegraphics[width=.8\maxwidth]{#1}}
    % Ensure that by default, figures have no caption (until we provide a
    % proper Figure object with a Caption API and a way to capture that
    % in the conversion process - todo).
    \usepackage{caption}
    \DeclareCaptionLabelFormat{nolabel}{}
    \captionsetup{labelformat=nolabel}

    \usepackage{adjustbox} % Used to constrain images to a maximum size 
    \usepackage{xcolor} % Allow colors to be defined
    \usepackage{enumerate} % Needed for markdown enumerations to work
    \usepackage{geometry} % Used to adjust the document margins
    \usepackage{amsmath} % Equations
    \usepackage{amssymb} % Equations
    \usepackage{textcomp} % defines textquotesingle
    % Hack from http://tex.stackexchange.com/a/47451/13684:
    \AtBeginDocument{%
        \def\PYZsq{\textquotesingle}% Upright quotes in Pygmentized code
    }
    \usepackage{upquote} % Upright quotes for verbatim code
    \usepackage{eurosym} % defines \euro
    \usepackage[mathletters]{ucs} % Extended unicode (utf-8) support
    \usepackage[utf8x]{inputenc} % Allow utf-8 characters in the tex document
    \usepackage{fancyvrb} % verbatim replacement that allows latex
    \usepackage{grffile} % extends the file name processing of package graphics 
                         % to support a larger range 
    % The hyperref package gives us a pdf with properly built
    % internal navigation ('pdf bookmarks' for the table of contents,
    % internal cross-reference links, web links for URLs, etc.)
    \usepackage{hyperref}
    \usepackage{longtable} % longtable support required by pandoc >1.10
    \usepackage{booktabs}  % table support for pandoc > 1.12.2
    \usepackage[inline]{enumitem} % IRkernel/repr support (it uses the enumerate* environment)
    \usepackage[normalem]{ulem} % ulem is needed to support strikethroughs (\sout)
                                % normalem makes italics be italics, not underlines
    \usepackage{mathrsfs}
    

    
    
    % Colors for the hyperref package
    \definecolor{urlcolor}{rgb}{0,.145,.698}
    \definecolor{linkcolor}{rgb}{.71,0.21,0.01}
    \definecolor{citecolor}{rgb}{.12,.54,.11}

    % ANSI colors
    \definecolor{ansi-black}{HTML}{3E424D}
    \definecolor{ansi-black-intense}{HTML}{282C36}
    \definecolor{ansi-red}{HTML}{E75C58}
    \definecolor{ansi-red-intense}{HTML}{B22B31}
    \definecolor{ansi-green}{HTML}{00A250}
    \definecolor{ansi-green-intense}{HTML}{007427}
    \definecolor{ansi-yellow}{HTML}{DDB62B}
    \definecolor{ansi-yellow-intense}{HTML}{B27D12}
    \definecolor{ansi-blue}{HTML}{208FFB}
    \definecolor{ansi-blue-intense}{HTML}{0065CA}
    \definecolor{ansi-magenta}{HTML}{D160C4}
    \definecolor{ansi-magenta-intense}{HTML}{A03196}
    \definecolor{ansi-cyan}{HTML}{60C6C8}
    \definecolor{ansi-cyan-intense}{HTML}{258F8F}
    \definecolor{ansi-white}{HTML}{C5C1B4}
    \definecolor{ansi-white-intense}{HTML}{A1A6B2}
    \definecolor{ansi-default-inverse-fg}{HTML}{FFFFFF}
    \definecolor{ansi-default-inverse-bg}{HTML}{000000}

    % commands and environments needed by pandoc snippets
    % extracted from the output of `pandoc -s`
    \providecommand{\tightlist}{%
      \setlength{\itemsep}{0pt}\setlength{\parskip}{0pt}}
    \DefineVerbatimEnvironment{Highlighting}{Verbatim}{commandchars=\\\{\}}
    % Add ',fontsize=\small' for more characters per line
    \newenvironment{Shaded}{}{}
    \newcommand{\KeywordTok}[1]{\textcolor[rgb]{0.00,0.44,0.13}{\textbf{{#1}}}}
    \newcommand{\DataTypeTok}[1]{\textcolor[rgb]{0.56,0.13,0.00}{{#1}}}
    \newcommand{\DecValTok}[1]{\textcolor[rgb]{0.25,0.63,0.44}{{#1}}}
    \newcommand{\BaseNTok}[1]{\textcolor[rgb]{0.25,0.63,0.44}{{#1}}}
    \newcommand{\FloatTok}[1]{\textcolor[rgb]{0.25,0.63,0.44}{{#1}}}
    \newcommand{\CharTok}[1]{\textcolor[rgb]{0.25,0.44,0.63}{{#1}}}
    \newcommand{\StringTok}[1]{\textcolor[rgb]{0.25,0.44,0.63}{{#1}}}
    \newcommand{\CommentTok}[1]{\textcolor[rgb]{0.38,0.63,0.69}{\textit{{#1}}}}
    \newcommand{\OtherTok}[1]{\textcolor[rgb]{0.00,0.44,0.13}{{#1}}}
    \newcommand{\AlertTok}[1]{\textcolor[rgb]{1.00,0.00,0.00}{\textbf{{#1}}}}
    \newcommand{\FunctionTok}[1]{\textcolor[rgb]{0.02,0.16,0.49}{{#1}}}
    \newcommand{\RegionMarkerTok}[1]{{#1}}
    \newcommand{\ErrorTok}[1]{\textcolor[rgb]{1.00,0.00,0.00}{\textbf{{#1}}}}
    \newcommand{\NormalTok}[1]{{#1}}
    
    % Additional commands for more recent versions of Pandoc
    \newcommand{\ConstantTok}[1]{\textcolor[rgb]{0.53,0.00,0.00}{{#1}}}
    \newcommand{\SpecialCharTok}[1]{\textcolor[rgb]{0.25,0.44,0.63}{{#1}}}
    \newcommand{\VerbatimStringTok}[1]{\textcolor[rgb]{0.25,0.44,0.63}{{#1}}}
    \newcommand{\SpecialStringTok}[1]{\textcolor[rgb]{0.73,0.40,0.53}{{#1}}}
    \newcommand{\ImportTok}[1]{{#1}}
    \newcommand{\DocumentationTok}[1]{\textcolor[rgb]{0.73,0.13,0.13}{\textit{{#1}}}}
    \newcommand{\AnnotationTok}[1]{\textcolor[rgb]{0.38,0.63,0.69}{\textbf{\textit{{#1}}}}}
    \newcommand{\CommentVarTok}[1]{\textcolor[rgb]{0.38,0.63,0.69}{\textbf{\textit{{#1}}}}}
    \newcommand{\VariableTok}[1]{\textcolor[rgb]{0.10,0.09,0.49}{{#1}}}
    \newcommand{\ControlFlowTok}[1]{\textcolor[rgb]{0.00,0.44,0.13}{\textbf{{#1}}}}
    \newcommand{\OperatorTok}[1]{\textcolor[rgb]{0.40,0.40,0.40}{{#1}}}
    \newcommand{\BuiltInTok}[1]{{#1}}
    \newcommand{\ExtensionTok}[1]{{#1}}
    \newcommand{\PreprocessorTok}[1]{\textcolor[rgb]{0.74,0.48,0.00}{{#1}}}
    \newcommand{\AttributeTok}[1]{\textcolor[rgb]{0.49,0.56,0.16}{{#1}}}
    \newcommand{\InformationTok}[1]{\textcolor[rgb]{0.38,0.63,0.69}{\textbf{\textit{{#1}}}}}
    \newcommand{\WarningTok}[1]{\textcolor[rgb]{0.38,0.63,0.69}{\textbf{\textit{{#1}}}}}
    
    
    % Define a nice break command that doesn't care if a line doesn't already
    % exist.
    \def\br{\hspace*{\fill} \\* }
    % Math Jax compatibility definitions
    \def\gt{>}
    \def\lt{<}
    \let\Oldtex\TeX
    \let\Oldlatex\LaTeX
    \renewcommand{\TeX}{\textrm{\Oldtex}}
    \renewcommand{\LaTeX}{\textrm{\Oldlatex}}
    % Document parameters
    % Document title
    \title{PS2}
    
    
    
    
    

    % Pygments definitions
    
\makeatletter
\def\PY@reset{\let\PY@it=\relax \let\PY@bf=\relax%
    \let\PY@ul=\relax \let\PY@tc=\relax%
    \let\PY@bc=\relax \let\PY@ff=\relax}
\def\PY@tok#1{\csname PY@tok@#1\endcsname}
\def\PY@toks#1+{\ifx\relax#1\empty\else%
    \PY@tok{#1}\expandafter\PY@toks\fi}
\def\PY@do#1{\PY@bc{\PY@tc{\PY@ul{%
    \PY@it{\PY@bf{\PY@ff{#1}}}}}}}
\def\PY#1#2{\PY@reset\PY@toks#1+\relax+\PY@do{#2}}

\expandafter\def\csname PY@tok@w\endcsname{\def\PY@tc##1{\textcolor[rgb]{0.73,0.73,0.73}{##1}}}
\expandafter\def\csname PY@tok@c\endcsname{\let\PY@it=\textit\def\PY@tc##1{\textcolor[rgb]{0.25,0.50,0.50}{##1}}}
\expandafter\def\csname PY@tok@cp\endcsname{\def\PY@tc##1{\textcolor[rgb]{0.74,0.48,0.00}{##1}}}
\expandafter\def\csname PY@tok@k\endcsname{\let\PY@bf=\textbf\def\PY@tc##1{\textcolor[rgb]{0.00,0.50,0.00}{##1}}}
\expandafter\def\csname PY@tok@kp\endcsname{\def\PY@tc##1{\textcolor[rgb]{0.00,0.50,0.00}{##1}}}
\expandafter\def\csname PY@tok@kt\endcsname{\def\PY@tc##1{\textcolor[rgb]{0.69,0.00,0.25}{##1}}}
\expandafter\def\csname PY@tok@o\endcsname{\def\PY@tc##1{\textcolor[rgb]{0.40,0.40,0.40}{##1}}}
\expandafter\def\csname PY@tok@ow\endcsname{\let\PY@bf=\textbf\def\PY@tc##1{\textcolor[rgb]{0.67,0.13,1.00}{##1}}}
\expandafter\def\csname PY@tok@nb\endcsname{\def\PY@tc##1{\textcolor[rgb]{0.00,0.50,0.00}{##1}}}
\expandafter\def\csname PY@tok@nf\endcsname{\def\PY@tc##1{\textcolor[rgb]{0.00,0.00,1.00}{##1}}}
\expandafter\def\csname PY@tok@nc\endcsname{\let\PY@bf=\textbf\def\PY@tc##1{\textcolor[rgb]{0.00,0.00,1.00}{##1}}}
\expandafter\def\csname PY@tok@nn\endcsname{\let\PY@bf=\textbf\def\PY@tc##1{\textcolor[rgb]{0.00,0.00,1.00}{##1}}}
\expandafter\def\csname PY@tok@ne\endcsname{\let\PY@bf=\textbf\def\PY@tc##1{\textcolor[rgb]{0.82,0.25,0.23}{##1}}}
\expandafter\def\csname PY@tok@nv\endcsname{\def\PY@tc##1{\textcolor[rgb]{0.10,0.09,0.49}{##1}}}
\expandafter\def\csname PY@tok@no\endcsname{\def\PY@tc##1{\textcolor[rgb]{0.53,0.00,0.00}{##1}}}
\expandafter\def\csname PY@tok@nl\endcsname{\def\PY@tc##1{\textcolor[rgb]{0.63,0.63,0.00}{##1}}}
\expandafter\def\csname PY@tok@ni\endcsname{\let\PY@bf=\textbf\def\PY@tc##1{\textcolor[rgb]{0.60,0.60,0.60}{##1}}}
\expandafter\def\csname PY@tok@na\endcsname{\def\PY@tc##1{\textcolor[rgb]{0.49,0.56,0.16}{##1}}}
\expandafter\def\csname PY@tok@nt\endcsname{\let\PY@bf=\textbf\def\PY@tc##1{\textcolor[rgb]{0.00,0.50,0.00}{##1}}}
\expandafter\def\csname PY@tok@nd\endcsname{\def\PY@tc##1{\textcolor[rgb]{0.67,0.13,1.00}{##1}}}
\expandafter\def\csname PY@tok@s\endcsname{\def\PY@tc##1{\textcolor[rgb]{0.73,0.13,0.13}{##1}}}
\expandafter\def\csname PY@tok@sd\endcsname{\let\PY@it=\textit\def\PY@tc##1{\textcolor[rgb]{0.73,0.13,0.13}{##1}}}
\expandafter\def\csname PY@tok@si\endcsname{\let\PY@bf=\textbf\def\PY@tc##1{\textcolor[rgb]{0.73,0.40,0.53}{##1}}}
\expandafter\def\csname PY@tok@se\endcsname{\let\PY@bf=\textbf\def\PY@tc##1{\textcolor[rgb]{0.73,0.40,0.13}{##1}}}
\expandafter\def\csname PY@tok@sr\endcsname{\def\PY@tc##1{\textcolor[rgb]{0.73,0.40,0.53}{##1}}}
\expandafter\def\csname PY@tok@ss\endcsname{\def\PY@tc##1{\textcolor[rgb]{0.10,0.09,0.49}{##1}}}
\expandafter\def\csname PY@tok@sx\endcsname{\def\PY@tc##1{\textcolor[rgb]{0.00,0.50,0.00}{##1}}}
\expandafter\def\csname PY@tok@m\endcsname{\def\PY@tc##1{\textcolor[rgb]{0.40,0.40,0.40}{##1}}}
\expandafter\def\csname PY@tok@gh\endcsname{\let\PY@bf=\textbf\def\PY@tc##1{\textcolor[rgb]{0.00,0.00,0.50}{##1}}}
\expandafter\def\csname PY@tok@gu\endcsname{\let\PY@bf=\textbf\def\PY@tc##1{\textcolor[rgb]{0.50,0.00,0.50}{##1}}}
\expandafter\def\csname PY@tok@gd\endcsname{\def\PY@tc##1{\textcolor[rgb]{0.63,0.00,0.00}{##1}}}
\expandafter\def\csname PY@tok@gi\endcsname{\def\PY@tc##1{\textcolor[rgb]{0.00,0.63,0.00}{##1}}}
\expandafter\def\csname PY@tok@gr\endcsname{\def\PY@tc##1{\textcolor[rgb]{1.00,0.00,0.00}{##1}}}
\expandafter\def\csname PY@tok@ge\endcsname{\let\PY@it=\textit}
\expandafter\def\csname PY@tok@gs\endcsname{\let\PY@bf=\textbf}
\expandafter\def\csname PY@tok@gp\endcsname{\let\PY@bf=\textbf\def\PY@tc##1{\textcolor[rgb]{0.00,0.00,0.50}{##1}}}
\expandafter\def\csname PY@tok@go\endcsname{\def\PY@tc##1{\textcolor[rgb]{0.53,0.53,0.53}{##1}}}
\expandafter\def\csname PY@tok@gt\endcsname{\def\PY@tc##1{\textcolor[rgb]{0.00,0.27,0.87}{##1}}}
\expandafter\def\csname PY@tok@err\endcsname{\def\PY@bc##1{\setlength{\fboxsep}{0pt}\fcolorbox[rgb]{1.00,0.00,0.00}{1,1,1}{\strut ##1}}}
\expandafter\def\csname PY@tok@kc\endcsname{\let\PY@bf=\textbf\def\PY@tc##1{\textcolor[rgb]{0.00,0.50,0.00}{##1}}}
\expandafter\def\csname PY@tok@kd\endcsname{\let\PY@bf=\textbf\def\PY@tc##1{\textcolor[rgb]{0.00,0.50,0.00}{##1}}}
\expandafter\def\csname PY@tok@kn\endcsname{\let\PY@bf=\textbf\def\PY@tc##1{\textcolor[rgb]{0.00,0.50,0.00}{##1}}}
\expandafter\def\csname PY@tok@kr\endcsname{\let\PY@bf=\textbf\def\PY@tc##1{\textcolor[rgb]{0.00,0.50,0.00}{##1}}}
\expandafter\def\csname PY@tok@bp\endcsname{\def\PY@tc##1{\textcolor[rgb]{0.00,0.50,0.00}{##1}}}
\expandafter\def\csname PY@tok@fm\endcsname{\def\PY@tc##1{\textcolor[rgb]{0.00,0.00,1.00}{##1}}}
\expandafter\def\csname PY@tok@vc\endcsname{\def\PY@tc##1{\textcolor[rgb]{0.10,0.09,0.49}{##1}}}
\expandafter\def\csname PY@tok@vg\endcsname{\def\PY@tc##1{\textcolor[rgb]{0.10,0.09,0.49}{##1}}}
\expandafter\def\csname PY@tok@vi\endcsname{\def\PY@tc##1{\textcolor[rgb]{0.10,0.09,0.49}{##1}}}
\expandafter\def\csname PY@tok@vm\endcsname{\def\PY@tc##1{\textcolor[rgb]{0.10,0.09,0.49}{##1}}}
\expandafter\def\csname PY@tok@sa\endcsname{\def\PY@tc##1{\textcolor[rgb]{0.73,0.13,0.13}{##1}}}
\expandafter\def\csname PY@tok@sb\endcsname{\def\PY@tc##1{\textcolor[rgb]{0.73,0.13,0.13}{##1}}}
\expandafter\def\csname PY@tok@sc\endcsname{\def\PY@tc##1{\textcolor[rgb]{0.73,0.13,0.13}{##1}}}
\expandafter\def\csname PY@tok@dl\endcsname{\def\PY@tc##1{\textcolor[rgb]{0.73,0.13,0.13}{##1}}}
\expandafter\def\csname PY@tok@s2\endcsname{\def\PY@tc##1{\textcolor[rgb]{0.73,0.13,0.13}{##1}}}
\expandafter\def\csname PY@tok@sh\endcsname{\def\PY@tc##1{\textcolor[rgb]{0.73,0.13,0.13}{##1}}}
\expandafter\def\csname PY@tok@s1\endcsname{\def\PY@tc##1{\textcolor[rgb]{0.73,0.13,0.13}{##1}}}
\expandafter\def\csname PY@tok@mb\endcsname{\def\PY@tc##1{\textcolor[rgb]{0.40,0.40,0.40}{##1}}}
\expandafter\def\csname PY@tok@mf\endcsname{\def\PY@tc##1{\textcolor[rgb]{0.40,0.40,0.40}{##1}}}
\expandafter\def\csname PY@tok@mh\endcsname{\def\PY@tc##1{\textcolor[rgb]{0.40,0.40,0.40}{##1}}}
\expandafter\def\csname PY@tok@mi\endcsname{\def\PY@tc##1{\textcolor[rgb]{0.40,0.40,0.40}{##1}}}
\expandafter\def\csname PY@tok@il\endcsname{\def\PY@tc##1{\textcolor[rgb]{0.40,0.40,0.40}{##1}}}
\expandafter\def\csname PY@tok@mo\endcsname{\def\PY@tc##1{\textcolor[rgb]{0.40,0.40,0.40}{##1}}}
\expandafter\def\csname PY@tok@ch\endcsname{\let\PY@it=\textit\def\PY@tc##1{\textcolor[rgb]{0.25,0.50,0.50}{##1}}}
\expandafter\def\csname PY@tok@cm\endcsname{\let\PY@it=\textit\def\PY@tc##1{\textcolor[rgb]{0.25,0.50,0.50}{##1}}}
\expandafter\def\csname PY@tok@cpf\endcsname{\let\PY@it=\textit\def\PY@tc##1{\textcolor[rgb]{0.25,0.50,0.50}{##1}}}
\expandafter\def\csname PY@tok@c1\endcsname{\let\PY@it=\textit\def\PY@tc##1{\textcolor[rgb]{0.25,0.50,0.50}{##1}}}
\expandafter\def\csname PY@tok@cs\endcsname{\let\PY@it=\textit\def\PY@tc##1{\textcolor[rgb]{0.25,0.50,0.50}{##1}}}

\def\PYZbs{\char`\\}
\def\PYZus{\char`\_}
\def\PYZob{\char`\{}
\def\PYZcb{\char`\}}
\def\PYZca{\char`\^}
\def\PYZam{\char`\&}
\def\PYZlt{\char`\<}
\def\PYZgt{\char`\>}
\def\PYZsh{\char`\#}
\def\PYZpc{\char`\%}
\def\PYZdl{\char`\$}
\def\PYZhy{\char`\-}
\def\PYZsq{\char`\'}
\def\PYZdq{\char`\"}
\def\PYZti{\char`\~}
% for compatibility with earlier versions
\def\PYZat{@}
\def\PYZlb{[}
\def\PYZrb{]}
\makeatother


    % Exact colors from NB
    \definecolor{incolor}{rgb}{0.0, 0.0, 0.5}
    \definecolor{outcolor}{rgb}{0.545, 0.0, 0.0}



    
    % Prevent overflowing lines due to hard-to-break entities
    \sloppy 
    % Setup hyperref package
    \hypersetup{
      breaklinks=true,  % so long urls are correctly broken across lines
      colorlinks=true,
      urlcolor=urlcolor,
      linkcolor=linkcolor,
      citecolor=citecolor,
      }
    % Slightly bigger margins than the latex defaults
    
    \geometry{verbose,tmargin=1in,bmargin=1in,lmargin=1in,rmargin=1in}
    
    

    \begin{document}
    
    
    \maketitle
    
    

    
    \hypertarget{problem-set-2}{%
\subsubsection{Problem Set 2}\label{problem-set-2}}

\textbf{Made By:} Eric Englin, Eric Berube, Gabriella Pierre

    \hypertarget{problem-1}{%
\subsubsection{Problem \#1}\label{problem-1}}

    \textbf{A)} Arteaga sought to find out if the return to college
education lies within the human capital method, where going to more
school and more classes would have impact on income and skill, or if it
was part of the signaling method which would show that the people that
got into Los Andes would have made that income whether they had extra
classes or not.

    \textbf{B)} In this study Arteaga uses the difference-in-difference
approach to compare the wages earned by Los Andes graduates in the years
before and after the policy change with other similar top 10 school
graduates and their earned wages. Since the graduates from all schools
were applying for the same jobs in the same sectors the only change in
hiring trends and wages should be a result of the Los Andes curriculum
change. She found that after the change the Los Andes graduates did
worse on recruitment exams and were hired up to 17\% less than before
the change.

\textbf{Equation:}
\(wage_{it} = \beta_0 + \beta_1 Andes_i*Post_t + \beta_2 Andes_i + \beta_3 Post_t + \beta_4 experience_{i,t} + \epsilon_{it}\)

    \textbf{C)}

\emph{Variable meanings} 1. \(wage_{it}\): wages earned 2. \(Andes\):
flag to designate if person went to Universidad de los Andes 3.
\(Post\): lag that this is in the post-curriculum change period 4.
\(experience\): years since graduation 5. \(\epsilon\): error term

    \textbf{D)} Beta 1 on Andes in the Post period because we want to see
how the curriculum change affects wages. This change is only seen in
post-andes beta1.

\textbf{E)} The human capital model is strongly supported. She found
that after the change the Los Andes graduates did worse on recruitment
exams and were hired up to 17\% less than before the change.

\textbf{F)} Regression discontinuity uses arbitrary cutoffs by looking
at individuals or samples right above and below this cutoff to find the
average treatment effect. This has been used to find the effect of
certain colleges that have a cutoff for test scores or GPAs.
Thistlethwaite used this design to understand the effect of the National
Merit Scholarship program by looking at students who were around the
cutoff for receiving and not receiving the award.

    \hypertarget{problem-2}{%
\subsubsection{Problem 2}\label{problem-2}}

\textbf{A)} Read the paper!

\textbf{B)} This is measured in 3 ways: 1. \textbf{SAT average for
admissions}: This can be a fairly problematic indicator because it is
going to favor richer and less diverse students, which biases our
quality indicator. It would be beneficial if they had some measure to
indicate the output, rather than inputs like test scores. 2.
\textbf{Barron's index}: This may be a better indicator for signaling,
but it seems fairly good for determining quality of colleges at a
national level. 3. \textbf{Net tuition cost}: Not a good measure because
many expensive schools cater to a different kind of experience that may
be based around learning, but can also be based around having a quality
``life experience'' or lifestyle while on campus.

\textbf{C)} Beta = Holding quality measure 2 constant, every 1 point
increase in quality measure 1 is related to a beta increase in log
earnings. Theta is the same, but holding beta constant and using quality
measure 2.

\textbf{D)} The major omitted variable is based around parental income.
These incomes were missing in many cases, so they built a regression to
engineer features for the parental income using occupation \& education
data available for student families. You definitely want to account for
this, so its a good solution to do this procedure to minimize error
rate. However, more accurate and consistent data around income would
always be helpful. If the data is not available in any capacity, I would
think they could predict family income a bit better if they also
factored in location and demographic data along with occupation and
education levels.

    \begin{Verbatim}[commandchars=\\\{\}]
{\color{incolor}In [{\color{incolor}3}]:} \PY{c+c1}{\PYZsh{}import library}
        \PY{k+kn}{import} \PY{n+nn}{pandas} \PY{k}{as} \PY{n+nn}{pd}
        
        \PY{c+c1}{\PYZsh{}import data}
        \PY{n}{df} \PY{o}{=} \PY{n}{pd}\PY{o}{.}\PY{n}{read\PYZus{}excel}\PY{p}{(}\PY{l+s+sa}{r}\PY{l+s+s2}{\PYZdq{}}\PY{l+s+s2}{./data/PS2\PYZus{}tables.xlsx}\PY{l+s+s2}{\PYZdq{}}\PY{p}{,}\PY{n}{sheet\PYZus{}name}\PY{o}{=}\PY{l+s+s2}{\PYZdq{}}\PY{l+s+s2}{Problem2}\PY{l+s+s2}{\PYZdq{}}\PY{p}{)}
        
        \PY{c+c1}{\PYZsh{}create empty datafields for confidence intervals}
        \PY{n}{df}\PY{p}{[}\PY{l+s+s1}{\PYZsq{}}\PY{l+s+s1}{CI}\PY{l+s+s1}{\PYZsq{}}\PY{p}{]}\PY{o}{=}\PY{l+s+s2}{\PYZdq{}}\PY{l+s+s2}{\PYZdq{}}
        \PY{n}{df}\PY{p}{[}\PY{l+s+s1}{\PYZsq{}}\PY{l+s+s1}{CI\PYZus{}noHBCU}\PY{l+s+s1}{\PYZsq{}}\PY{p}{]}\PY{o}{=}\PY{l+s+s2}{\PYZdq{}}\PY{l+s+s2}{\PYZdq{}}
        
        \PY{c+c1}{\PYZsh{}loop through each measure (SAT, tuition, Barron) and method (Basic, Self\PYZhy{}revelation)}
        \PY{k}{for} \PY{n}{x} \PY{o+ow}{in} \PY{n}{df}\PY{o}{.}\PY{n}{index}\PY{p}{:}
            \PY{n}{change} \PY{o}{=} \PY{n+nb}{round}\PY{p}{(}\PY{l+m+mi}{2}\PY{o}{*}\PY{n}{df}\PY{p}{[}\PY{l+s+s1}{\PYZsq{}}\PY{l+s+s1}{StDev}\PY{l+s+s1}{\PYZsq{}}\PY{p}{]}\PY{p}{[}\PY{n}{x}\PY{p}{]}\PY{p}{,}\PY{l+m+mi}{2}\PY{p}{)} \PY{c+c1}{\PYZsh{}twice the standard deviation}
            \PY{n}{coef} \PY{o}{=} \PY{n}{df}\PY{p}{[}\PY{l+s+s1}{\PYZsq{}}\PY{l+s+s1}{Coefficient}\PY{l+s+s1}{\PYZsq{}}\PY{p}{]}\PY{p}{[}\PY{n}{x}\PY{p}{]} \PY{c+c1}{\PYZsh{} field coefficient}
            \PY{n}{df}\PY{p}{[}\PY{l+s+s1}{\PYZsq{}}\PY{l+s+s1}{CI}\PY{l+s+s1}{\PYZsq{}}\PY{p}{]}\PY{p}{[}\PY{n}{x}\PY{p}{]}\PY{o}{=}\PY{l+s+s1}{\PYZsq{}}\PY{l+s+s1}{[}\PY{l+s+s1}{\PYZsq{}}\PY{o}{+}\PY{n+nb}{str}\PY{p}{(}\PY{n+nb}{round}\PY{p}{(}\PY{n}{coef}\PY{o}{\PYZhy{}}\PY{n}{change}\PY{p}{,}\PY{l+m+mi}{3}\PY{p}{)}\PY{p}{)}\PY{o}{+}\PY{l+s+s1}{\PYZsq{}}\PY{l+s+s1}{, }\PY{l+s+s1}{\PYZsq{}}\PY{o}{+}\PY{p}{(}\PY{n+nb}{str}\PY{p}{(}\PY{n+nb}{round}\PY{p}{(}\PY{n}{coef}\PY{o}{+}\PY{n}{change}\PY{p}{,}\PY{l+m+mi}{3}\PY{p}{)}\PY{p}{)}\PY{p}{)}\PY{o}{+}\PY{l+s+s1}{\PYZsq{}}\PY{l+s+s1}{]}\PY{l+s+s1}{\PYZsq{}} 
        
            \PY{n}{change2} \PY{o}{=} \PY{n+nb}{round}\PY{p}{(}\PY{l+m+mi}{2}\PY{o}{*}\PY{n}{df}\PY{p}{[}\PY{l+s+s1}{\PYZsq{}}\PY{l+s+s1}{StDev\PYZus{}noHBCU}\PY{l+s+s1}{\PYZsq{}}\PY{p}{]}\PY{p}{[}\PY{n}{x}\PY{p}{]}\PY{p}{,}\PY{l+m+mi}{2}\PY{p}{)} \PY{c+c1}{\PYZsh{}twice the standard deviation}
            \PY{n}{coef2} \PY{o}{=} \PY{n}{df}\PY{p}{[}\PY{l+s+s1}{\PYZsq{}}\PY{l+s+s1}{Coefficient\PYZus{}noHBCU}\PY{l+s+s1}{\PYZsq{}}\PY{p}{]}\PY{p}{[}\PY{n}{x}\PY{p}{]} \PY{c+c1}{\PYZsh{} field coefficient}
            \PY{n}{df}\PY{p}{[}\PY{l+s+s1}{\PYZsq{}}\PY{l+s+s1}{CI\PYZus{}noHBCU}\PY{l+s+s1}{\PYZsq{}}\PY{p}{]}\PY{p}{[}\PY{n}{x}\PY{p}{]}\PY{o}{=}\PY{p}{(}\PY{l+s+s1}{\PYZsq{}}\PY{l+s+s1}{[}\PY{l+s+s1}{\PYZsq{}}\PY{o}{+}\PY{n+nb}{str}\PY{p}{(}\PY{n+nb}{round}\PY{p}{(}\PY{n}{coef2}\PY{o}{\PYZhy{}}\PY{n}{change2}\PY{p}{,}\PY{l+m+mi}{3}\PY{p}{)}\PY{p}{)}\PY{o}{+}\PY{l+s+s1}{\PYZsq{}}\PY{l+s+s1}{, }\PY{l+s+s1}{\PYZsq{}}
                                    \PY{o}{+}\PY{p}{(}\PY{n+nb}{str}\PY{p}{(}\PY{n+nb}{round}\PY{p}{(}\PY{n}{coef2}\PY{o}{+}\PY{n}{change2}\PY{p}{,}\PY{l+m+mi}{3}\PY{p}{)}\PY{p}{)}\PY{p}{)}\PY{o}{+}\PY{l+s+s1}{\PYZsq{}}\PY{l+s+s1}{]}\PY{l+s+s1}{\PYZsq{}}\PY{p}{)}
\end{Verbatim}

    \textbf{Table for all black \& hispanic students:}

\begin{longtable}[]{@{}llll@{}}
\toprule
Field & Coefficient & StDev & CI\tabularnewline
\midrule
\endhead
Basic* (SAT Score) & 0.067 & 0.019 & {[}0.027, 0.107{]}\tabularnewline
Self-Revelation* (SAT Score) & 0.076 & 0.032 & {[}0.016,
0.136{]}\tabularnewline
Basic* (Net Tuition) & 0.173 & 0.056 & {[}0.063, 0.283{]}\tabularnewline
Self-Revelation (Net Tuition) & 0.138 & 0.071 & {[}-0.002,
0.278{]}\tabularnewline
Basic* (Barron's Index) & 0.063 & 0.022 & {[}0.023,
0.103{]}\tabularnewline
Self-Revelation (Barron's Index) & 0.049 & 0.036 & {[}-0.021,
0.119{]}\tabularnewline
\bottomrule
\end{longtable}

\textbf{Table for all black \& hispanic students, excluding HBCUs:}

\begin{longtable}[]{@{}llll@{}}
\toprule
Field & Coefficient & StDev & CI\tabularnewline
\midrule
\endhead
Basic* (SAT Score) & 0.122 & 0.030 & {[}0.062, 0.182{]}\tabularnewline
Self-Revelation* (SAT Score) & 0.120 & 0.042 & {[}0.04,
0.2{]}\tabularnewline
Basic* (Net Tuition) & 0.187 & 0.064 & {[}0.057, 0.317{]}\tabularnewline
Self-Revelation (Net Tuition) & 0.166 & 0.079 & {[}0.006,
0.326{]}\tabularnewline
Basic* (Barron's Index) & 0.158 & 0.040 & {[}0.078,
0.238{]}\tabularnewline
Self-Revelation (Barron's Index) & 0.143 & 0.053 & {[}0.033,
0.253{]}\tabularnewline
\bottomrule
\end{longtable}

    \textbf{F)} Once you look at the self-revelation model, the college
effect on student wages becomes statistically similar to 0. However,
this trend isn't seen among Black and Hispanic students excluding HBCUs.
Among these students, the wages increase by all metrics used.

\textbf{G)} This paper shows that a higher quality college has a higher
impact on black and hispanic students future wages, so these colleges,
including Harvard, should take this as an opportunity to help increase
future wages for these groups.

    \hypertarget{problem-3}{%
\subsubsection{Problem \#3}\label{problem-3}}

\textbf{A)} College educated workers are affected because there are
spillover effects and complementary jobs needed to help the higher skill
workers (example: assistant makes the business person more
effective/productive). College educated workers help other high-skill
workers when the spillover effect is greater than the decrease in
private returns due to more competition in the labor market.

\textbf{B)} Our spillovers: - \textbf{Eric Berube}: When I was in the
Army there was a noticeable spillover in terms of jobs created when I
was assigned to different units. I went to certain schools that resulted
in certifications and authorization to perform specific roles as an
officer. By holding those certifications my unit was allowed to assign
more soldiers (high sc hool educated) to work underneath me in key
positions. By getting those extra soldiers that freed up work hours from
other people who had been covering those responsibilities. There was a
compound effect from my arrival to work efficiency and positions
available. Having another manager also meant that my boss (college
educated) could focus more on their specific tasks instead of using
their work week to supervise the other people in the unit. -
\textbf{Gabriella Pierre}: As a soon to be worker in an educational
non-profit committed to advancing equity for low-income, minoritized
students, the spillovers I create on workers with less than a college
degree are increased access to opportunities regardless of their
educational background as I advocate for their inclusion in the work
that we do. For college-educated workers, I am causing those individuals
who may work with me, to be more productive and perform better,
increasing overall team success. - \textbf{Eric Englin}: Before coming
to grad school, I worked with homeless service providers to help get
people into housing. One big part of my job was communicating to
legislators that we were spending money effectively, so I think that a
potential spillover would be if I did a good job in communicating that
we needed more money to end homelessness, this could pave the way for
more funding to all positions of people working in the same area. It
would effectively increase the size of the pot for everyone.

\textbf{C)}

This is the table showing the current working population and the future
working population assuming that this program increases the college
educated group by 1\%, from 291 at 31\% to 298 at 32\%.

\begin{longtable}[]{@{}lllll@{}}
\toprule
\begin{minipage}[b]{0.21\columnwidth}\raggedright
Education Level\strut
\end{minipage} & \begin{minipage}[b]{0.23\columnwidth}\raggedright
Total (thousands)\strut
\end{minipage} & \begin{minipage}[b]{0.15\columnwidth}\raggedright
\% of total\strut
\end{minipage} & \begin{minipage}[b]{0.15\columnwidth}\raggedright
Total Under New Program (thousands)\strut
\end{minipage} & \begin{minipage}[b]{0.13\columnwidth}\raggedright
\% of total\strut
\end{minipage}\tabularnewline
\midrule
\endhead
\begin{minipage}[t]{0.21\columnwidth}\raggedright
Less than HS\strut
\end{minipage} & \begin{minipage}[t]{0.23\columnwidth}\raggedright
81\strut
\end{minipage} & \begin{minipage}[t]{0.15\columnwidth}\raggedright
9\%\strut
\end{minipage} & \begin{minipage}[t]{0.15\columnwidth}\raggedright
81\strut
\end{minipage} & \begin{minipage}[t]{0.13\columnwidth}\raggedright
9\%\strut
\end{minipage}\tabularnewline
\begin{minipage}[t]{0.21\columnwidth}\raggedright
HS Grad\strut
\end{minipage} & \begin{minipage}[t]{0.23\columnwidth}\raggedright
309\strut
\end{minipage} & \begin{minipage}[t]{0.15\columnwidth}\raggedright
33\%\strut
\end{minipage} & \begin{minipage}[t]{0.15\columnwidth}\raggedright
309\strut
\end{minipage} & \begin{minipage}[t]{0.13\columnwidth}\raggedright
33\%\strut
\end{minipage}\tabularnewline
\begin{minipage}[t]{0.21\columnwidth}\raggedright
Some College\strut
\end{minipage} & \begin{minipage}[t]{0.23\columnwidth}\raggedright
250\strut
\end{minipage} & \begin{minipage}[t]{0.15\columnwidth}\raggedright
27\%\strut
\end{minipage} & \begin{minipage}[t]{0.15\columnwidth}\raggedright
250\strut
\end{minipage} & \begin{minipage}[t]{0.13\columnwidth}\raggedright
27\%\strut
\end{minipage}\tabularnewline
\begin{minipage}[t]{0.21\columnwidth}\raggedright
College\strut
\end{minipage} & \begin{minipage}[t]{0.23\columnwidth}\raggedright
291\strut
\end{minipage} & \begin{minipage}[t]{0.15\columnwidth}\raggedright
31\%\strut
\end{minipage} & \begin{minipage}[t]{0.15\columnwidth}\raggedright
298\strut
\end{minipage} & \begin{minipage}[t]{0.13\columnwidth}\raggedright
32\%\strut
\end{minipage}\tabularnewline
\begin{minipage}[t]{0.21\columnwidth}\raggedright
Total\strut
\end{minipage} & \begin{minipage}[t]{0.23\columnwidth}\raggedright
931\strut
\end{minipage} & \begin{minipage}[t]{0.15\columnwidth}\raggedright
\strut
\end{minipage} & \begin{minipage}[t]{0.15\columnwidth}\raggedright
938\strut
\end{minipage} & \begin{minipage}[t]{0.13\columnwidth}\raggedright
\strut
\end{minipage}\tabularnewline
\bottomrule
\end{longtable}

\emph{NPV Calculation:}

\begin{itemize}
\tightlist
\item
  Increased Income: \$ 7,000\$ people \(x 55,640\) (Annaul Wages)
  \(= 389,480,000\)
\item
  Program Cost: \(-\$70,000,000\)
\item
  \(NPV = -\$70,000,000 + 389,480,000 + \frac{389,480,000}{(1+0.03)^1} + \frac{389,480,000}{(1+0.03)^2} + \frac{389,480,000}{(1+0.03)^3}\)
\item
  \(NPV = -70,000,000 + 389,480,000 + 378,135,922 + 367,122,255\) 
\item
  \(NPV = \$1,064,738,177\)
\end{itemize}

\textbf{D)} An instrumental variable is needed to determine the causal
effect of this policy. This means that we need to show that the social
returns are increased solely due to the 93 additional college educated
workers coming into Maine. There are many unobservable factors that
could skew our effect and instrumental variables help with this.

\textbf{E)} We need to find an instrumental variable that is relevant
(i.e.~correlated with the outcome of social return to having more
college educated workers) and exclusionary (i.e.~uncorrelated with
omitted variables).

\textbf{F)} To control for sources of potential bias, Moretti used
Census data to produce two instrumental variables: (1) the lagged city
demographic structure and (2) the presence of a land--grant college. We
would use the land-grant cities as an instrumental variable because
Maine has 2 land-grant universities. This would provide a fairly
interpretable outcome that could be used to explain the gains from the
policy to a wide audience group.

\textbf{G)} For our analysis, we are looking at the land grant analyses
and will base our predictions off of the 1990 findings. With the
increases in years 2 \& 3, we found a total NPV increase in salaries to
be \textbf{\$579,314,072.60}. These are shown in the table \&
calculations below:

\begin{longtable}[]{@{}llllllll@{}}
\toprule
\begin{minipage}[b]{0.13\columnwidth}\raggedright
Education Level\strut
\end{minipage} & \begin{minipage}[b]{0.14\columnwidth}\raggedright
Percent Increase\strut
\end{minipage} & \begin{minipage}[b]{0.09\columnwidth}\raggedright
Starting Salary\strut
\end{minipage} & \begin{minipage}[b]{0.09\columnwidth}\raggedright
Year 2 \& 3 Salary\strut
\end{minipage} & \begin{minipage}[b]{0.08\columnwidth}\raggedright
Salary Difference\strut
\end{minipage} & \begin{minipage}[b]{0.08\columnwidth}\raggedright
Total NPV per Person\strut
\end{minipage} & \begin{minipage}[b]{0.08\columnwidth}\raggedright
Total People\strut
\end{minipage} & \begin{minipage}[b]{0.08\columnwidth}\raggedright
Total NPV\strut
\end{minipage}\tabularnewline
\midrule
\endhead
\begin{minipage}[t]{0.13\columnwidth}\raggedright
Less than HS\strut
\end{minipage} & \begin{minipage}[t]{0.14\columnwidth}\raggedright
0.77\%\strut
\end{minipage} & \begin{minipage}[t]{0.09\columnwidth}\raggedright
\$26,350\strut
\end{minipage} & \begin{minipage}[t]{0.09\columnwidth}\raggedright
\$26,552\strut
\end{minipage} & \begin{minipage}[t]{0.08\columnwidth}\raggedright
\$202.90\strut
\end{minipage} & \begin{minipage}[t]{0.08\columnwidth}\raggedright
\$376.93\strut
\end{minipage} & \begin{minipage}[t]{0.08\columnwidth}\raggedright
81,000\strut
\end{minipage} & \begin{minipage}[t]{0.08\columnwidth}\raggedright
\$30,530,978\strut
\end{minipage}\tabularnewline
\begin{minipage}[t]{0.13\columnwidth}\raggedright
HS Grad\strut
\end{minipage} & \begin{minipage}[t]{0.14\columnwidth}\raggedright
0.84\%\strut
\end{minipage} & \begin{minipage}[t]{0.09\columnwidth}\raggedright
\$37,120\strut
\end{minipage} & \begin{minipage}[t]{0.09\columnwidth}\raggedright
\$37,431\strut
\end{minipage} & \begin{minipage}[t]{0.08\columnwidth}\raggedright
\$311.81\strut
\end{minipage} & \begin{minipage}[t]{0.08\columnwidth}\raggedright
\$579.26\strut
\end{minipage} & \begin{minipage}[t]{0.08\columnwidth}\raggedright
309,000\strut
\end{minipage} & \begin{minipage}[t]{0.08\columnwidth}\raggedright
\$178,990,547\strut
\end{minipage}\tabularnewline
\begin{minipage}[t]{0.13\columnwidth}\raggedright
Some College\strut
\end{minipage} & \begin{minipage}[t]{0.14\columnwidth}\raggedright
0.94\%\strut
\end{minipage} & \begin{minipage}[t]{0.09\columnwidth}\raggedright
\$46,810\strut
\end{minipage} & \begin{minipage}[t]{0.09\columnwidth}\raggedright
\$47,250\strut
\end{minipage} & \begin{minipage}[t]{0.08\columnwidth}\raggedright
\$440.01\strut
\end{minipage} & \begin{minipage}[t]{0.08\columnwidth}\raggedright
\$817.43\strut
\end{minipage} & \begin{minipage}[t]{0.08\columnwidth}\raggedright
250,000\strut
\end{minipage} & \begin{minipage}[t]{0.08\columnwidth}\raggedright
\$204,357,634\strut
\end{minipage}\tabularnewline
\begin{minipage}[t]{0.13\columnwidth}\raggedright
College\strut
\end{minipage} & \begin{minipage}[t]{0.14\columnwidth}\raggedright
0.55\%\strut
\end{minipage} & \begin{minipage}[t]{0.09\columnwidth}\raggedright
\$55,640\strut
\end{minipage} & \begin{minipage}[t]{0.09\columnwidth}\raggedright
\$55,946\strut
\end{minipage} & \begin{minipage}[t]{0.08\columnwidth}\raggedright
\$306.03\strut
\end{minipage} & \begin{minipage}[t]{0.08\columnwidth}\raggedright
\$568.50\strut
\end{minipage} & \begin{minipage}[t]{0.08\columnwidth}\raggedright
291,000\strut
\end{minipage} & \begin{minipage}[t]{0.08\columnwidth}\raggedright
\$165,434,911\strut
\end{minipage}\tabularnewline
\begin{minipage}[t]{0.13\columnwidth}\raggedright
\textbf{Spillover Total}\strut
\end{minipage} & \begin{minipage}[t]{0.14\columnwidth}\raggedright
-\strut
\end{minipage} & \begin{minipage}[t]{0.09\columnwidth}\raggedright
-\strut
\end{minipage} & \begin{minipage}[t]{0.09\columnwidth}\raggedright
-\strut
\end{minipage} & \begin{minipage}[t]{0.08\columnwidth}\raggedright
-\strut
\end{minipage} & \begin{minipage}[t]{0.08\columnwidth}\raggedright
\$2,342.12\strut
\end{minipage} & \begin{minipage}[t]{0.08\columnwidth}\raggedright
931,000\strut
\end{minipage} & \begin{minipage}[t]{0.08\columnwidth}\raggedright
\textbf{\$579,314,072}\strut
\end{minipage}\tabularnewline
\bottomrule
\end{longtable}

\emph{Calculations for NPV}

\begin{itemize}
\item
  \(Wages_{starting} * Percent Increase = Wages_{Year2}= Wages_{Year3}\)
\item
  \(Wages_{Year2or3} - Wages_{starting} = Wages_{difference}\)
\item
  \(NPV_{Person} = \frac{Wages_{difference}}{(1+r)^2}+\frac{Wages_{difference}}{(1+r)^3}\)

  \begin{itemize}
  \tightlist
  \item
    Note: \(Wages_{difference} = 0\) in year 1 because effect of program
    isn't seen until year 2
  \end{itemize}
\item
  \(NPV_{total} = NPV_{per person} * Workers\)
\end{itemize}

\emph{Calculations shown for Less than HS:}

\begin{itemize}
\item
  Percent increase taken from Morettie (2002 paper) as 0.77 for every 1
  percentage point increase in college educated workers

  \begin{itemize}
  \item
    \(\$26,350 * 1.0077 = \$26,552\)
  \item
    \(\$26,552 - \$26,350 = \$202.90\)
  \item
    \(NPV = \frac{\$202.90}{(1+0.03)^2}+\frac{\$202.90}{(1+0.03)^3}\)
  \item
    \(NPV = \$191.25 + \$185.68\)
  \item
    \(NPV = \$376.93\) per person
  \item
    \(NPV_{total} = \$376.93\) per person \(* 81,000\) people
    \(= \$30,530,978.78\)
  \end{itemize}
\end{itemize}

\textbf{H)} Shown in the table below, the tax revenue from the social
spillover among all existing groups is \(\$40,551,985.08\). However, an
additional \(\$104,892.07\) will be gained through tax revenue from the
\(7,000\) new college educated workers (for simplicity, our assumption
is that these workers would not see their wages don't increase over
time). This would bring the total to \(\$91,949,100.74\). From all of
this income, it would likely pay for itself as a program.

\begin{longtable}[]{@{}llllll@{}}
\toprule
\begin{minipage}[b]{0.19\columnwidth}\raggedright
Education Level\strut
\end{minipage} & \begin{minipage}[b]{0.14\columnwidth}\raggedright
Year 2 \& 3 Salary Difference\strut
\end{minipage} & \begin{minipage}[b]{0.13\columnwidth}\raggedright
Total NPV per Person\strut
\end{minipage} & \begin{minipage}[b]{0.13\columnwidth}\raggedright
Total People\strut
\end{minipage} & \begin{minipage}[b]{0.13\columnwidth}\raggedright
Total NPV\strut
\end{minipage} & \begin{minipage}[b]{0.13\columnwidth}\raggedright
Tax Revenue\strut
\end{minipage}\tabularnewline
\midrule
\endhead
\begin{minipage}[t]{0.19\columnwidth}\raggedright
Less than HS\strut
\end{minipage} & \begin{minipage}[t]{0.14\columnwidth}\raggedright
\$202.90\strut
\end{minipage} & \begin{minipage}[t]{0.13\columnwidth}\raggedright
\$376.93\strut
\end{minipage} & \begin{minipage}[t]{0.13\columnwidth}\raggedright
81,000\strut
\end{minipage} & \begin{minipage}[t]{0.13\columnwidth}\raggedright
\$30,530,978\strut
\end{minipage} & \begin{minipage}[t]{0.13\columnwidth}\raggedright
\$2,137,168\strut
\end{minipage}\tabularnewline
\begin{minipage}[t]{0.19\columnwidth}\raggedright
HS Grad\strut
\end{minipage} & \begin{minipage}[t]{0.14\columnwidth}\raggedright
\$311.81\strut
\end{minipage} & \begin{minipage}[t]{0.13\columnwidth}\raggedright
\$579.26\strut
\end{minipage} & \begin{minipage}[t]{0.13\columnwidth}\raggedright
309,000\strut
\end{minipage} & \begin{minipage}[t]{0.13\columnwidth}\raggedright
\$178,990,547\strut
\end{minipage} & \begin{minipage}[t]{0.13\columnwidth}\raggedright
\$12,529,338\strut
\end{minipage}\tabularnewline
\begin{minipage}[t]{0.19\columnwidth}\raggedright
Some College\strut
\end{minipage} & \begin{minipage}[t]{0.14\columnwidth}\raggedright
\$440.01\strut
\end{minipage} & \begin{minipage}[t]{0.13\columnwidth}\raggedright
\$817.43\strut
\end{minipage} & \begin{minipage}[t]{0.13\columnwidth}\raggedright
250,000\strut
\end{minipage} & \begin{minipage}[t]{0.13\columnwidth}\raggedright
\$204,357,634\strut
\end{minipage} & \begin{minipage}[t]{0.13\columnwidth}\raggedright
\$14,305,034\strut
\end{minipage}\tabularnewline
\begin{minipage}[t]{0.19\columnwidth}\raggedright
College\strut
\end{minipage} & \begin{minipage}[t]{0.14\columnwidth}\raggedright
\$306.03\strut
\end{minipage} & \begin{minipage}[t]{0.13\columnwidth}\raggedright
\$568.50\strut
\end{minipage} & \begin{minipage}[t]{0.13\columnwidth}\raggedright
291,000\strut
\end{minipage} & \begin{minipage}[t]{0.13\columnwidth}\raggedright
\$165,434,911\strut
\end{minipage} & \begin{minipage}[t]{0.13\columnwidth}\raggedright
\$11,580,443\strut
\end{minipage}\tabularnewline
\begin{minipage}[t]{0.19\columnwidth}\raggedright
\textbf{Spillover Total}\strut
\end{minipage} & \begin{minipage}[t]{0.14\columnwidth}\raggedright
-\strut
\end{minipage} & \begin{minipage}[t]{0.13\columnwidth}\raggedright
\$2,342.12\strut
\end{minipage} & \begin{minipage}[t]{0.13\columnwidth}\raggedright
931,000\strut
\end{minipage} & \begin{minipage}[t]{0.13\columnwidth}\raggedright
\textbf{\$579,314,072}\strut
\end{minipage} & \begin{minipage}[t]{0.13\columnwidth}\raggedright
\textbf{\$40,551,985}\strut
\end{minipage}\tabularnewline
\begin{minipage}[t]{0.19\columnwidth}\raggedright
--\strut
\end{minipage} & \begin{minipage}[t]{0.14\columnwidth}\raggedright
--\strut
\end{minipage} & \begin{minipage}[t]{0.13\columnwidth}\raggedright
--\strut
\end{minipage} & \begin{minipage}[t]{0.13\columnwidth}\raggedright
--\strut
\end{minipage} & \begin{minipage}[t]{0.13\columnwidth}\raggedright
--\strut
\end{minipage} & \begin{minipage}[t]{0.13\columnwidth}\raggedright
--\strut
\end{minipage}\tabularnewline
\begin{minipage}[t]{0.19\columnwidth}\raggedright
New College Workers\strut
\end{minipage} & \begin{minipage}[t]{0.14\columnwidth}\raggedright
\$55,640\strut
\end{minipage} & \begin{minipage}[t]{0.13\columnwidth}\raggedright
\$104,892.07\strut
\end{minipage} & \begin{minipage}[t]{0.13\columnwidth}\raggedright
7,000\strut
\end{minipage} & \begin{minipage}[t]{0.13\columnwidth}\raggedright
\$734,244,509\strut
\end{minipage} & \begin{minipage}[t]{0.13\columnwidth}\raggedright
\$51,397,115\strut
\end{minipage}\tabularnewline
\begin{minipage}[t]{0.19\columnwidth}\raggedright
\textbf{Spillover Plus New Total}\strut
\end{minipage} & \begin{minipage}[t]{0.14\columnwidth}\raggedright
-\strut
\end{minipage} & \begin{minipage}[t]{0.13\columnwidth}\raggedright
\$107,234.19\strut
\end{minipage} & \begin{minipage}[t]{0.13\columnwidth}\raggedright
938,000\strut
\end{minipage} & \begin{minipage}[t]{0.13\columnwidth}\raggedright
\$1,313,558,581\strut
\end{minipage} & \begin{minipage}[t]{0.13\columnwidth}\raggedright
\textbf{\$91,949,100}\strut
\end{minipage}\tabularnewline
\bottomrule
\end{longtable}

\textbf{I)} The tax revenue from the social spillover among all existing
groups is \(\$40,551,985.08\). However, an additional
\(\$51,397,115.66\) will be gained through tax revenue from the
\(9,300\) new college educated workers (assuming their wages don't
increase over time). This would bring the total to \(\$91,949,100.74\).
If the total program cost is \$10,000 per person, this would be a \$70
million program that would have a net benefit of \(\$21,949,100.74\).

\textbf{J)} Wages will increase the most for less educated workers,
which would decrease wage inequality.


    % Add a bibliography block to the postdoc
    
    
    
    \end{document}
